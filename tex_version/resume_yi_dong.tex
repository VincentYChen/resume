% possible options include font size ('10pt', '11pt' and '12pt'), 
% paper size ('a4paper', 'letterpaper', 'a5paper', 'legalpaper', 'executivepaper' and 'landscape') 
% and font family ('sans' and 'roman')
\documentclass[10pt,a4paper,roman]{moderncv}

% moderncv 主题
\moderncvstyle{classic}                       % 选项参数是 ‘casual’, ‘classic’, ‘oldstyle’ 和 ’banking’
\moderncvcolor{blue}                          % 选项参数是 ‘blue’ (默认)、‘orange’、‘green’、‘red’、‘purple’ 和 ‘grey’
%\nopagenumbers{}                             % 消除注释以取消自动页码生成功能

% 调整页边距
\usepackage[scale=0.9]{geometry}
%\setlength{\hintscolumnwidth}{3cm}           % 如果你希望改变日期栏的宽度

% 字符编码
\usepackage{fontspec}
\usepackage{xunicode}
\usepackage{xeCJK}
\setmainfont{Courier New}%英文字体
\setsansfont{Microsoft YaHei}
\setmonofont{Microsoft YaHei}
\setCJKmainfont[Scale=0.9]{Microsoft YaHei}%中文字体
\setCJKmonofont[Scale=0.9]{Microsoft YaHei}
\setCJKfamilyfont{mono}{Microsoft YaHei}

%\hypersetup{unicode=true}
\geometry{a4paper, textwidth=6.5in, textheight=10in,marginparsep=7pt, marginparwidth=.6in}

% 个人信息
\name{祎东}{陈}
\title{C++软件工程师}                     % 可选项、如不需要可删除本行
\phone[mobile]{+86~(130)~2216~8160}                   % optional, remove / comment the line if not wanted; the optional "type" of the phone can be
\email{vincent.yd.chen@gmail.com}                               % optional, remove / comment the line if not wanted
\social[linkedin]{vincentychen}                        % optional, remove / comment the line if not wanted
\social[github]{VincentYChen}                              % optional, remove / comment the line if not wanted
\photo[64pt][0.4pt]{pic/vincent}                       % optional, remove / comment the line if not wanted; '64pt' is the height the picture must be resized to, 0.4pt is the thickness of the frame around it (put it to 0pt for no frame) and 'picture' is the name of the picture file

% 显示索引号;仅用于在简历中使用了引言
%\makeatletter
%\renewcommand*{\bibliographyitemlabel}{\@biblabel{\arabic{enumiv}}}
%\makeatother

% 分类索引
%\usepackage{multibib}
%\newcites{book,misc}{{Books},{Others}}
%----------------------------------------------------------------------------------
%            内容
%----------------------------------------------------------------------------------

\begin{document}
\makecvtitle

\section{教育背景}
\cventry{年 -- 年}{学位}{院校}{城市}{\textit{成绩}}{说明}  % 第3到第6编码可留白
\cventry{年 -- 年}{学位}{院校}{城市}{\textit{成绩}}{说明}

\section{毕业论文}
\cvitem{题目}{\emph{题目}}
\cvitem{导师}{导师}
\cvitem{说明}{\small 论文简介}
\section{工作背景}
\subsection{专业}
\cventry{年 -- 年}{职位}{公司}{城市}{}{不超过1--2行的概况说明\newline{}%
工作内容:%
\begin{itemize}%
\item 工作内容 1;
\item 工作内容 2、 含二级内容:
  \begin{itemize}%
  \item 二级内容 (a);
  \item 二级内容 (b)、含三级内容 (不建议使用);
    \begin{itemize}
    \item 三级内容 i;
    \item 三级内容 ii;
    \item 三级内容 iii;
    \end{itemize}
  \item 二级内容 (c);
  \end{itemize}
\item 工作内容 3。
\end{itemize}}
\cventry{年 -- 年}{职位}{公司}{城市}{}{说明行1\newline{}说明行2}
\subsection{其他}
\cventry{年 -- 年}{职位}{公司}{城市}{}{说明}

\section{语言技能}
\cvitemwithcomment{语言 1}{水平}{评价}
\cvitemwithcomment{语言 2}{水平}{评价}
\cvitemwithcomment{语言 3}{水平}{评价}

\section{计算机技能}
\cvdoubleitem{类别 1}{XXX, YYY, ZZZ}{类别 4}{XXX, YYY, ZZZ}
\cvdoubleitem{类别 2}{XXX, YYY, ZZZ}{类别 5}{XXX, YYY, ZZZ}
\cvdoubleitem{类别 3}{XXX, YYY, ZZZ}{类别 6}{XXX, YYY, ZZZ}

\section{个人兴趣}
\cvitem{爱好 1}{\small 说明}
\cvitem{爱好 2}{\small 说明}
\cvitem{爱好 3}{\small 说明}

\section{其他 1}
\cvlistitem{项目 1}
\cvlistitem{项目 2}
\cvlistitem{项目 3}

\renewcommand{\listitemsymbol}{-}             % 改变列表符号

\section{其他 2}
\cvlistdoubleitem{项目 1}{项目 4}
\cvlistdoubleitem{项目 2}{项目 5}
\cvlistdoubleitem{项目 3}{}


\end{document}